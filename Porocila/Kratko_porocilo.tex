% Options for packages loaded elsewhere
\PassOptionsToPackage{unicode}{hyperref}
\PassOptionsToPackage{hyphens}{url}
%
\documentclass[
]{article}
\usepackage{amsmath,amssymb}
\usepackage{lmodern}
\usepackage{iftex}
\ifPDFTeX
  \usepackage[T1]{fontenc}
  \usepackage[utf8]{inputenc}
  \usepackage{textcomp} % provide euro and other symbols
\else % if luatex or xetex
  \usepackage{unicode-math}
  \defaultfontfeatures{Scale=MatchLowercase}
  \defaultfontfeatures[\rmfamily]{Ligatures=TeX,Scale=1}
\fi
% Use upquote if available, for straight quotes in verbatim environments
\IfFileExists{upquote.sty}{\usepackage{upquote}}{}
\IfFileExists{microtype.sty}{% use microtype if available
  \usepackage[]{microtype}
  \UseMicrotypeSet[protrusion]{basicmath} % disable protrusion for tt fonts
}{}
\makeatletter
\@ifundefined{KOMAClassName}{% if non-KOMA class
  \IfFileExists{parskip.sty}{%
    \usepackage{parskip}
  }{% else
    \setlength{\parindent}{0pt}
    \setlength{\parskip}{6pt plus 2pt minus 1pt}}
}{% if KOMA class
  \KOMAoptions{parskip=half}}
\makeatother
\usepackage{xcolor}
\IfFileExists{xurl.sty}{\usepackage{xurl}}{} % add URL line breaks if available
\IfFileExists{bookmark.sty}{\usepackage{bookmark}}{\usepackage{hyperref}}
\hypersetup{
  pdftitle={Kratko porocilo},
  hidelinks,
  pdfcreator={LaTeX via pandoc}}
\urlstyle{same} % disable monospaced font for URLs
\usepackage[margin=1in]{geometry}
\usepackage{graphicx}
\makeatletter
\def\maxwidth{\ifdim\Gin@nat@width>\linewidth\linewidth\else\Gin@nat@width\fi}
\def\maxheight{\ifdim\Gin@nat@height>\textheight\textheight\else\Gin@nat@height\fi}
\makeatother
% Scale images if necessary, so that they will not overflow the page
% margins by default, and it is still possible to overwrite the defaults
% using explicit options in \includegraphics[width, height, ...]{}
\setkeys{Gin}{width=\maxwidth,height=\maxheight,keepaspectratio}
% Set default figure placement to htbp
\makeatletter
\def\fps@figure{htbp}
\makeatother
\setlength{\emergencystretch}{3em} % prevent overfull lines
\providecommand{\tightlist}{%
  \setlength{\itemsep}{0pt}\setlength{\parskip}{0pt}}
\setcounter{secnumdepth}{-\maxdimen} % remove section numbering
\ifLuaTeX
  \usepackage{selnolig}  % disable illegal ligatures
\fi

\title{Kratko porocilo}
\author{}
\date{\vspace{-2.5em}2022-03-24}

\begin{document}
\maketitle

\hypertarget{opis-projekta}{%
\subsection{Opis projekta}\label{opis-projekta}}

Analizirala bom Nashov model pogajanja pri prenosljivih dobrinah.
Gledala bom izplačila v matrični igri za dva igralca. Rešitev je vedno
sporazum. Razlikovala bom med enostopensko in dvostopensko igro. Razlika
je v tem katero točko vzamemo za status quo. Pri reševanju sporazuma
izhajamo iz strateške igre za dva igralca. Na začetku imamo torej dve
matriki koristnosti A in B. Nashov model pogajanja pri prenosljivih
dobrinah je ekvivalenten igri z mešanimi strategijami, kjer ima prvi
igralec koristi v matriki A-B, drugi pa A-B transponirano. To je torej
matrična igra, pri kateri matriko sestavljajo razlike A-B.

\hypertarget{plan-dela}{%
\subsection{Plan dela}\label{plan-dela}}

\hypertarget{dosedanje-delo}{%
\section{Dosedanje delo}\label{dosedanje-delo}}

Projekt delam v programu R. Kot prvo sem definirala enostopensko in
dvostopensko igro pogajanja. Pri enostopenski sem si pomagala s funkcijo
gt\_nbs iz \(library(hop)\). Pri dvostopenski pa sem status quo določila
s pomočjo linearne regresije. Tako sem definirala funkcije minmax\_p in
minmax\_q, ki vrneta najboljši strategiji akcij prvega in drugega
igralca. Funkciji enofazno\_pogajanje in dvofazno\_pogajanje pa vrneta
vektor izplačil obeh igralcev pri sporazumu.

Odločila sem se, da bom primerjala igro pogajanja glede na 4
porazdelitve. To so
\[N(3, 0.7), \quad Beta(5,1), \quad Invgama(2, 0.5), \quad Exp(3).\]
Funkcija, ki generira eno od teh matrik je funkcija \(matrika.\) Pri
vsaki igri bomo potrebovali dve različni, za prvega in za drugega
igralca.

Težave sem imela pri funkcijah za eno in dvostopensko pogajanje, saj sem
prvo uporabila že vgrajeno funkcijo \(gt_minmax\). Le ta je prav tako
vrgla najboljše strategije, vendar je delovala le za matrike do
velikosti \(4 x 4\). V kolikor je bila večja je potrebovala veliko časa
in prostora. Kot drugo pa sem imela nekaj težavo tudi s teorijo, kako je
pravilni izračun točke sporazuma. Kot ugotavljam pa mi kljub popravku
funkcije maxmin še vedno vrača napačne rezultate.

Analizirala bom velikost izplačil posameznega igralca. Gledala bom
različne velikosti matrik, max velikost bo 40 x 40. Matrike bodo velike,
zato bo treba vrednosti izplačil shraniti v RDS datoteke.

\hypertarget{nadaljevanje-dela}{%
\section{Nadaljevanje dela}\label{nadaljevanje-dela}}

Analiza podatkov glede na: \(\dot\) Prvi igralec ima neko fiksno
porazdelitev in kako porazdelitev akcij drugega igralca vpliva na
sporazum. \(\dot\) Kako velikost matrike vpliva na sporazum, pri fiksnih
matrikah A in B.

Celotna analiza bo predstavljena v aplikaciji, kjer bodo možnosti izbire
različne velikosti matrike koristnosti za posameznega igralca in njune
porazdelitve.

\end{document}
